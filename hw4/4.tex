\documentclass{article}
\usepackage{fontspec}
\usepackage{xcolor}

\usepackage{amsthm}
\usepackage{amsmath}
\usepackage{amssymb}
\usepackage{unicode-math}
\usepackage[makeroom]{cancel}

\usepackage[normalem]{ulem}

\setmainfont{Times New Roman}
\setmathfont{Latin Modern Math}

\setlength\parindent{0em}
\setlength\parskip{0.618em}
\usepackage[a4paper,lmargin=1in,rmargin=1in,tmargin=1in,bmargin=1in]{geometry}

\usepackage{enumitem}

\renewcommand\qedsymbol{$\blacksquare$}

\begin{document}

\begin{center}
  \textbf{MATH} 136---\textbf{HOMEWORK} 4

  \color{red}R\color{teal}icardo
  \color{red}J\color{cyan}.
  \color{red}A\color{teal}cu$\color{red}{\widetilde{\color{teal}\text{n}}}$\color{teal}a\color{black}

  \color{teal}(\color{red}862079740\color{teal})\color{black}
\end{center}\vspace{1.618em}

\paragraph{1} Find all the quadratic residues of 3

\begin{tabular}{|c|c|}
  \hline x&x^2 \\
  \hline
  0& 0\enskip \equiv \enskip 0(\text{mod} 3)\\
  1& 1\enskip \equiv \enskip 1(\text{mod} 3)\\
  2& 4\enskip \equiv \enskip 1(\text{mod} 3)\\ \hline
\end{tabular}

This table gives all the possible values for a square mod 3.

So the only quadratic residue of 3 is 1.


\paragraph{1} Find all the quadratic residues of 19

\begin{tabular}{|c|c|}
  \hline x&x^2 \\
  \hline
0 & 0 \enskip \equiv \enskip 0 (\text{mod} 3)\\
1 & 1 \enskip \equiv \enskip 1 (\text{mod} 3)\\
2 & 4 \enskip \equiv \enskip 4 (\text{mod} 3)\\
3 & 9 \enskip \equiv \enskip 9 (\text{mod} 3)\\
4 & 16 \enskip \equiv \enskip 16 (\text{mod} 3)\\
5 & 25 \enskip \equiv \enskip 6 (\text{mod} 3)\\
6 & 36 \enskip \equiv \enskip 17 (\text{mod} 3)\\
7 & 49 \enskip \equiv \enskip 11 (\text{mod} 3)\\
8 & 64 \enskip \equiv \enskip 7 (\text{mod} 3)\\
9 & 81 \enskip \equiv \enskip 5 (\text{mod} 3)\\
10 & 100 \enskip \equiv \enskip 5 (\text{mod} 3)\\
11 & 121 \enskip \equiv \enskip 7 (\text{mod} 3)\\
12 & 144 \enskip \equiv \enskip 11 (\text{mod} 3)\\
13 & 169 \enskip \equiv \enskip 17 (\text{mod} 3)\\
14 & 196 \enskip \equiv \enskip 6 (\text{mod} 3)\\
15 & 225 \enskip \equiv \enskip 16 (\text{mod} 3)\\
16 & 256 \enskip \equiv \enskip 9 (\text{mod} 3)\\
17 & 289 \enskip \equiv \enskip 4 (\text{mod} 3)\\
  18 & 324 \enskip \equiv \enskip 1 (\text{mod} 3)\\ \hline
\end{tabular}

This table gives all the possible values for a square mod 19.

So the quadratic residues of 19 are 1,4,5,6,7,9,11,16, and 17.

\color{red} Note: Jose said 0 is not considered a quadratic residue \color{black}
\newpage
\paragraph{3} Find all the values of Legendre symbol $\left(
  \frac{j}{7} \right)$ for $j = 1, 2, 3, 4, 5, 6$

\begin{tabular}{|c|c|}
  \hline x&x^2 \\
  \hline
0 & 0 \enskip \equiv \enskip 0 (\text{mod} 3)\\
1 & 1 \enskip \equiv \enskip 1 (\text{mod} 3)\\
2 & 4 \enskip \equiv \enskip 4 (\text{mod} 3)\\
3 & 9 \enskip \equiv \enskip 2 (\text{mod} 3)\\
4 & 16 \enskip \equiv \enskip 2 (\text{mod} 3)\\
5 & 25 \enskip \equiv \enskip 4 (\text{mod} 3)\\
6 & 36 \enskip \equiv \enskip 1 (\text{mod} 3)\\ \hline
\end{tabular}

So, reading the table of squares modulo 7 we have:

$\left(
  \frac{1}{7} \right) = \left(
  \frac{4}{7} \right) =
\left(\frac{2}{7} \right) = 1$ since they're all quadratic residues. And,
$\left(
  \frac{3}{7} \right) = \left(
  \frac{5}{7} \right) =\left(
  \frac{6}{7} \right) = -1$ since they're not quadratic residues.

\paragraph{4} Evaluate the Legendre symbol $\left( \frac{7}{11}
\right)$ by using Euler's criterion.

$\left( \frac{7}{11}
\right) = 7^{\frac{\phi(11)}{2}} (\text{mod} 11) \stackrel{11\text{ is
    prime }}{=} 7^{\frac{11-1}{2}} (\text{mod} 11)
= 7^5 (\text{mod} 11) = 16807 \equiv 10 =  10-11 \equiv -1 (\text{mod} 11)$

\paragraph{5}  Let $a$ and $b$ be integers not divisible by $p$. Show
that either one or all of the three integers $a$,$b$ and $ab$ are
quadratic residues of $p$.

\uwave{Slu\enskip.}
\vspace{0.618 em}

$p$ does not divide $a \implies \left( \frac{a}{p}
\right) \neq 0$

$\implies \left( \frac{a}{p}
\right) = 1$ or $\left( \frac{a}{p}
\right) = -1$

$p$ does not divide $b \implies \left( \frac{b}{p}
\right) \neq 0$

$\implies \left( \frac{b}{p}
\right) = 1$ or $\left( \frac{b}{p}
\right) = -1$


The evaluation of the possibilities for the Legendre symbol for $a$
and $b$ establishes are three possibilities for one of the pair $a$ and
$b$ being a quadratic residue of $p$. Either $a$ and $b$ are quadratic
residues of $p$. Either $a$ or $b$ is a quadratic residue of $p$. Or
neither $a$ nor $b$ are quadratic residues of $p$. In that case we
want to show $ab$ is a quadratic residue of $p$.

$\left( \frac{ab}{p}
\right) = \left( \frac{a}{p}
\right) \left( \frac{b}{p}
\right) = (-1)(-1) = 1$

So, either one or all of the three integers $a$,$b$ and $ab$ are
quadratic residues of $p$.

\vspace{0.618 em}
$\blacksquare$


\newpage
\paragraph{6} Let $p$ be a prime and $a$ be a quadratic residue of
$p$. Show that if $p \equiv 1 (\text{mod} 4)$, then $-a$ is also a
quadratic residue of $p$, whereas if $p\equiv 3 (\text{mod} 4)$, then
$-a$ is a quadratic nonresidue of $p$.

\uwave{Slu\enskip.}
\vspace{0.618 em}

$\left( \frac{-a}{p}
\right) = \left( \frac{(-1)(a)}{p}
\right)= \left( \frac{-1}{p}
\right) \left( \frac{a}{p}
\right) = \left( \frac{-1}{p}
\right)\cdot 1  = \left( \frac{-1}{p}
\right)$ Since $a$ is a quadratic residue of $p$.

For the first case:
$p \equiv 1 \enskip (\text{mod } 4) \implies p -1 \equiv 0 \enskip (\text{mod } 4)
\implies \exists k\in \mathbb{Z}: p-1 = 4k \implies \frac{p-1}{2} =
\frac{4k}{2} = 2k$

$\implies \frac{p-1}{2}$ is even

So, since $p$ is prime by Euler's criterion $\left( \frac{-a}{p}
\right) =\left( \frac{-1}{p}
\right) = (-1)^{\frac{p-1}{2}} = (-1)^{\text{even}} = 1 (\text{mod}
p)$.

So, $-a$ is
a quadratic residue modulo $p$.


For the first case:
$p \equiv 3 \enskip (\text{mod } 4) \implies p -1 \equiv 2 \enskip (\text{mod } 4)
\implies \exists k\in \mathbb{Z}: p-1 = 4k + 2 \implies \frac{p-1}{2} =
\frac{4k + 2}{2} = 2k +1$

$\implies \frac{p-1}{2}$ is odd

So, since $p$ is prime by Euler's criterion $\left( \frac{-a}{p}
\right) =\left( \frac{-1}{p}
\right) = (-1)^{\frac{p-1}{2}} = (-1)^{\text{odd}} = -1 (\text{mod}
p)$.

So, $-a$ is
a quadratic nonresidue modulo $p$.


\vspace{0.618 em}
$\blacksquare$

\paragraph{7} Show that if $p$ is an odd prime and $a$ is an integer
not divisible by $p$ then $\left( \frac{a^2}{p}
\right) = 1$.

\uwave{Slu\enskip.}
\vspace{0.618 em}


Since, either $a$ is a quadratic residue modulo $p$, and $a$ is not
divisible by $p$. We can say $\left( \frac{a}{p}
\right) = \pm 1$

$\left( \frac{a^2}{p}
\right) = \left( \frac{(a)(a)}{p}
\right)= \left( \frac{a}{p}
\right)\left( \frac{a}{p}
\right) = (\pm 1)^2 = 1 (\text{mod } p)$

So, if $p$ is an odd prime and $a$ is an integer
not divisible by $p$ then $\left( \frac{a^2}{p}
\right) = 1$.

\vspace{0.618 em}
$\blacksquare$


\end{document}
%%% Local Variables:
%%% mode: latex
%%% TeX-master: t
%%% End:
