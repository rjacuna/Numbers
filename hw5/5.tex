\documentclass{article}
\usepackage{fontspec}
\usepackage{xcolor}

\usepackage{amsthm}
\usepackage{amsmath}
\usepackage{amssymb}
\usepackage{unicode-math}
\usepackage[makeroom]{cancel}

\usepackage[normalem]{ulem}

\setmainfont{Times New Roman}
\setmathfont{Latin Modern Math}

\setlength\parindent{0em}
\setlength\parskip{0.618em}
\usepackage[a4paper,lmargin=1in,rmargin=1in,tmargin=1in,bmargin=1in]{geometry}

\usepackage{enumitem}

\renewcommand\qedsymbol{$\blacksquare$}

\begin{document}

\begin{center}
  \textbf{MATH} 136---\textbf{HOMEWORK} 5

  \color{red}R\color{teal}icardo
  \color{red}J\color{cyan}.
  \color{red}A\color{teal}cu$\color{red}{\widetilde{\color{teal}\text{n}}}$\color{teal}a\color{black}

  \color{teal} (\color{red}862079740\color{teal})\color{black}
\end{center}\vspace{1.618em}

\paragraph{1} Evaluate $(\frac{31}{641})$.

$\sqrt{31}\approx 5.5677643628300215 < 6$

So, we check if $31$ is divisible by every prime less than $6$:

$31 = 15\cdot 2 + 1\\
31 = 10\cdot 3 + 1\\
31 = 6\cdot 5 + 1$

No, so $31$ is prime.

$\sqrt{641} \approx 25.3179778023443 < 26$

So, we check if $641$ is divisible by every prime less than $26$:

$641 = 320\cdot2 + 1\\
641 = 213\cdot3 + 2\\
641 = 128\cdot5 + 1\\
641 = 91\cdot7 + 4\\
641 = 58\cdot11 + 3\\
641 = 49\cdot13 + 4\\
641 = 37\cdot17 + 12\\
641 = 33\cdot19 + 14\\
641 = 27\cdot23 + 20$

No, so $641$ is prime.

$(\frac{31}{641})(\frac{641}{31})=
(-1)^{\frac{31-1}{2}\frac{641-1}{2}}$

$\implies (\frac{31}{641})= (-1)^{\frac{31-1}{2}\frac{641-1}{2}}
(\frac{641}{31}) = (-1)^{15\cdot 320} (\frac{641}{31}) = (-1)^{2\cdot(15\cdot 160)}
(\frac{641}{31}) = (\frac{641}{31})$

$641 = 20\cdot 31 + 21 \implies (\frac{31}{641}) = (\frac{21}{31}) =
(\frac{3\cdot 7}{31}) = (\frac{3}{31}) (\frac{7}{31}) $

$(\frac{3}{31})  = (-1)^{\frac{3-1}{2}\frac{31-1}{2}}(\frac{31}{3}) =
(-1)^{1\cdot 15} (\frac{31}{3}) = -(\frac{31}{3})$

$31 = 10\cdot 3 + 1 \implies (\frac{3}{31}) = - (\frac{1}{3}) =
-(\frac{1^2}{3})=
-1 $

$(\frac{7}{31})  = (-1)^{\frac{7-1}{2}\frac{31-1}{2}}(\frac{31}{7}) =
(-1)^{3\cdot 15} (\frac{31}{7}) = -(\frac{31}{7})$

$31 = 4\cdot 7 + 3 \implies (\frac{31}{7}) = (\frac{3}{7}) \equiv
3^{\frac{7-1}{2}} \equiv 3^{3} \equiv 27 \equiv 3\cdot 7 + 6 \equiv 6
\equiv -1 ($ mod 7$)$

$\implies$
$(\frac{7}{31})  = -1\cdot -1 =1$

$\implies (\frac{31}{641}) = -1\cdot 1 = -1 $


\newpage
\paragraph{2} Show that if $p$ is an odd prime (bigger than $3$) then

\[\left(\frac{3}{p}\right) = \begin{cases} 1, $ if $ p\equiv \pm 1
    ($mod $ 12)\\ -1, $ if $ p\equiv \pm 5
    ($mod $ 12)
  \end{cases}\]

\uwave{pf\enskip.}
\vspace{0.618 em}

$\left(\frac{3}{p}\right) =
(-1)^{\frac{3-1}{2}\frac{p-1}{2}}\left(\frac{p}{3}\right)=
(-1)^{1\cdot \frac{p-1}{2}}\left(\frac{p}{3}\right)$
and Euler's Criterion

$\implies \left(\frac{3}{p}\right) =
(-1)^{\frac{p-1}{2}}\left(\frac{p}{3}\right) \equiv
(-1)^{\frac{p-1}{2}}p^{\frac{3-1}{2}} ($mod $3)\equiv
(-1)^{\frac{p-1}{2}}p^{1} ($mod $3)\equiv
(-1)^{\frac{p-1}{2}} p ($mod $3)$

We need to check the possibilities for the remainder
$r$, of $p$ when divided by $12$.

$\phi(12) = \phi(2^2\cdot 3) = \phi(2^2)\cdot \phi(3) =2^{2-1}(2-1)\cdot (3-1) = 4$

So, there are only $4$ numbers coprime to $12$.

We know $r \in \{0,1,2,3,4,5,6,7,8,9,10,11\}$. $r$ can't be $0$, since
$p$ is prime. $1^{\phi(12)}= 1 ($mod $12)$ so $1$ works. Now, $p$ is
prime so it's not divisible by $2$ or $3$, so $p$ can't be congruent
modulo $12$ to $4,6,8,$ and $10$. Now, we need to check the following
for completeness.

$5^{\phi(12)}= 5^4 = 625 = 52\cdot 12 + 1 \equiv 1 ($mod $12)$\\
$7^{\phi(12)}= 7^4 = 2401 = 200\cdot 12 + 1 \equiv 1 ($mod $12)$\\
$11^{\phi(12)}= 7^4 = 14641 = 1220\cdot 12 + 1 \equiv 1 ($mod $12)$.

$\phi(12) = 4$ together with $r^{\phi(12)}\equiv 1 ($mod $12) \iff
$gcd$(r,12) = 1$ tells us we are done. We found the four, relatively
prime numbers to $12$, that are less than $12$. And, any prime larger
than $3$ will be congruent to them.

Now $11 = 12 - 1 \equiv -1 ($mod $12)$ and $7 = 12 - 5 \equiv - 5 ($mod
$12)$

So $p$ is congruent to $\pm 1$ or $\pm 5$ modulo $12$.

$p \equiv 1 ($mod $12) \implies p = 12\cdot m + 1$

$\implies \left(\frac{3}{p}\right) \equiv (-1)^{\frac{12\cdot m +1 -1}{2}}
(12\cdot m +1)$
$\equiv (-1)^{2\cdot 3m}
(1)
\equiv 1 ($mod $3)$


$p \equiv -1 ($mod $12) \implies p = 12\cdot m - 1$

$\implies \left(\frac{3}{p}\right) \equiv (-1)^{\frac{12\cdot m -1 -1}{2}}
(12\cdot m -1)$
$\equiv (-1)^{6\cdot m -1}
(-1)
\equiv (-1)^{odd}(-1) \equiv 1 ($mod $3)$


$p \equiv 5 ($mod $12) \implies p = 12\cdot m + 5$

$\implies \left(\frac{3}{p}\right) \equiv (-1)^{\frac{12\cdot m +5 -1}{2}}
(12\cdot m +5)$
$\equiv (-1)^{6\cdot m + 2}
(5)
\equiv (-1)^{even}(-2\cdot 3 + 5) \equiv (1)(-1) \equiv -1 ($mod $3)$

$p \equiv -5 ($mod $12) \implies p = 12\cdot m - 5$

$\implies \left(\frac{3}{p}\right) \equiv (-1)^{\frac{12\cdot m -5 -1}{2}}
(12\cdot m -5)$
$\equiv (-1)^{6\cdot m - 3}
(-5)\\
\equiv (-1)^{3(2\cdot m - 1)}(2\cdot 3 - 5) \equiv (-1)^{odd}(1) \equiv -1 ($mod $3)$

So, if $p > 3$ and $p$ is prime.
\[\left(\frac{3}{p}\right) = \begin{cases} 1, $ if $ p\equiv \pm 1
    ($mod $ 12)\\ -1, $ if $ p\equiv \pm 5
    ($mod $ 12)
  \end{cases}\]

\vspace{0.618 em}
$\blacksquare$


\newpage
\paragraph{3} Find all positive integers $n$ for which $\phi(n) =
6$. Show that these are the only $n$ for which this holds.

\uwave{Pf\enskip.}
\vspace{0.618 em}

$\phi(n) = 6$ and $n \in \mathbb{N}$.

Either, $n$ is prime or not.

(I) $n$ is prime:

$\phi(n) = n-1 \implies n-1 = 6 \implies n = 7$

(II) $n$ isn't prime:

So, $n$ it's composite---i.e. $n = p_0^{m_0}
p_1^{m_1}...p_k^{m_k}$ where $p_i$ is prime, and $m_i\geq 0$, and $k$
finite.

$\phi(\cdot)$ is multiplicative.
$\implies \phi(n) = \phi(p_0^{m_0}
p_1^{m_1}...p_k^{m_k}) = p_0^{m_0-1}(p_0-1)
p_1^{m_1-1}(p_1-1)...p_k^{m_k-1}(p_k -1) = 6 = 2\cdot 3$

Part of the product corresponds to $2$ and part to $3$, since
$\phi(\cdot)$ is multiplicative.

So, let's consider up to rearrangement

$\phi(s) = p_0^{m_0-1}(p_0-1)...p_r^{m_r-1}(p_r-1) = 2$
and $\phi(t) = p_{r+1}^{m_{r+1}-1}(p_{r+1}-1)...p_{k}^{m_k-1}(p_k-1) = 3$


Since $2$ is prime, there's only one $p_i$ in the product and $([m_i -1=
0$ and $p_i-1= 2]$   or
$[p_i-1= 1$ and $p_i^{m_i-1} = 2])$

$m_i -1=
0$ and $p_i-1= 2 \implies p_i = 3 \implies s = 3$

$p_i-1= 1$ and $p_i^{m_i-1} = 2 \implies p_i = 2$ and
$2^{m_i-1}=2^1 \implies m_i-1 = 1 \implies m_i = 2 \implies s = 2^2 =4$

Of course we need to consider, values of $\phi(\cdot)$ where it is
equal to $1$.
$\phi(1)= 1,$ by convention.

For $p$ prime, $\phi(p) = 1\implies p-1 = 1 \implies p = 2.$

For a composite $l = q_0^{c_0}...q_a^{c_a}$, $\phi(l) =
q_0^{c_0-1}(q_0-1)...q_a^{c_a-1}(q_a-1)= 1$

All, the $c_i$ must be $1$, because if they weren't
their product of powers of primes would have a term bigger than $1$.

So, $\phi(l) = q_0^{0}(q_0-1)...q_a^{0}(q_a-1) = (q_0-1)...(q_a-1) =
1$

By the previous logic, one of the terms $q_i$ must be $2$. But, this
means that $1 = (q_0-1)...(2-1)...(q_a-1)$, that can't happen, as you'd have a product of primes bigger than $2$ minus
$1$, equaling $1$. So, $l$ isn't composite.


So, $\phi(3) = \phi(4) = \phi(3) \phi(2) = \phi(6) = 2$.

Note, $\phi(8) \neq \phi(4)\phi(2) = 2$ as gcd$(4,2) = 2$.

Since $3$ is prime, there's only one $p_j$ in the product\\ and $([m_j-1 =
0$ and $p_j - 1= 3]$   or
$[p_j-1= 1$ and $p_j^{m_j-1} = 3])$

$m_j =
0$ and $p_j - 1= 3 \implies p_j = 4$, but $p_j$ is prime so it can't
happen.

$p_j-1= 1$ and $p_j^{m_j-1} = 3 \implies p_j = 2 \implies 2^{m_j-1}=3$
is false, so $\not\exists n\in \mathbb{N}: \phi(n) = 3$

However, we have $\forall n,k \in \mathbb{N}: \phi(n^k)=
n^{k−1}\phi(n),$ where $n$ prime.

So, for each we can check,

$\phi(3) = 2 \implies 3\phi(3) = \phi(3^2) = \phi(9) = 3\cdot 2 =6$

$\phi(4) = 2$ gives no information as $4$ isn't a power of $3$.

And of course we can multiply by $2$ as gcd$(3,2) = 1$.

So, $\phi(18) = 6$

Finally, we need to consider multiplying by $2$ case (I).

This gives $\phi(14)=6$.

Summing up $n \in \{7,9,18,14\}\subset \mathbb{N}$, and the list is
exhaustive by the arguments above.

\vspace{0.618 em}
$\blacksquare$

\paragraph{4} Show that for $n$ a positive integer,

\[\phi(2n) = \begin{cases}\phi(n)$, if $n$ is odd$\\
                          2\phi(n)$, if $n$ is even$\\\end{cases}\]



\uwave{Pf\enskip.}
\vspace{0.618 em}

(I) $n$ is odd

$n$ is odd and $2$ is prime $\implies$ gcd$(n,2) = 1$

$\implies \phi(2n) = \phi(2)\phi(n) = (2-1)\phi(n) = \phi(n)$

(II) $n$ is even

$\implies \exists k\in \mathbb{Z}: n = 2^mk$, $m\geq 1$, and $k$ odd.

$\implies 2n = 2\cdot 2^mk = 2^{m+1}k$

$2$ is prime and $k$ is odd $\implies$ gcd$(k, 2^{m+1}) =1$

$\implies \phi(2^{m+1}k) = \phi(2^{m+1})\phi(k)$

$\forall n,k \in \mathbb{N}: \phi(n^k)=
n^{k−1}\phi(n),$ where $n$ prime.

$2$ is prime $\implies \phi(2^{m+1}) = 2^{m + 1-1}(2-1) = 2^m$

$\implies \phi(n) = 2^m\phi(k)$

If $m = 1$, we're done.

If $m > 1$, then $\phi(n) = 2\cdot 2^{m-1}(2-1)\phi(k) = 2
\cdot 2^{m-1}\phi(2)\phi(k) = 2\phi(2^m)\phi(k)$

$2$ is prime and $k$ is odd $\implies$ gcd$(k,2^m) = 1 \implies \phi(2^m)\phi(k) = \phi(2^mk) = \phi(n)$

$\implies \phi(n) = 2\phi(n)$

So,
\[\phi(2n) = \begin{cases}\phi(n)$, if $n$ is odd$\\
                          2\phi(n)$, if $n$ is even$\\\end{cases}\]
\vspace{0.618 em}
$\blacksquare$


\paragraph{5} Which positive integers have an odd number of positive
divisors? Explain why.

The squares. No clue. I just wrote this in python:

\begin{verbatim}
def divisors (x):
    return [y for y in range(2, x) if x%y == 0 ]

[x for x in range(1,10000) if not(len(divisors(x))%2 == 0) and
len(divisors(x)) >= 1]

=> [4,9,16,25,36,49,64,81,100,121,144,169,196,225,256,289,324,361,
    400,441,484,529,576,625,676,729,784,841,900,961,1024,1089,1156,
    1225,1296,1369,1444,1521,1600,1681,1764,1849,1936,2025,2116,
    2209,2304,2401,2500,2601,2704,2809,2916,3025,3136,3249,3364,
    3481,3600,3721,3844,3969,4096,4225,4356,4489,4624,4761,4900,
    5041,5184,5329,5476,5625,5776,5929,6084,6241,6400,6561,6724,
    6889,7056,7225,7396,7569,7744,7921,8100,8281,8464,8649,8836,
    9025,9216,9409,9604,9801]

\end{verbatim}






\end{document}
%%% Local Variables:
%%% mode: latex
%%% TeX-master: t
%%% End:
