\documentclass{article}
\usepackage{fontspec}
\usepackage{xcolor}

\usepackage{amsthm}
\usepackage{amsmath}
\usepackage{amssymb}
\usepackage{unicode-math}
\usepackage[makeroom]{cancel}

\usepackage[normalem]{ulem}

\setmainfont{Times New Roman}
\setmathfont{Latin Modern Math}

\setlength\parindent{0em}
\setlength\parskip{0.618em}
\usepackage[a4paper,lmargin=1in,rmargin=1in,tmargin=1in,bmargin=1in]{geometry}

\usepackage{enumitem}

\renewcommand\qedsymbol{$\blacksquare$}

\begin{document}

\begin{center}
  \textbf{MATH} 136---\textbf{HOMEWORK} 3

  \color{red}R\color{teal}icardo
  \color{red}J\color{cyan}.
  \color{red}A\color{teal}cu$\color{red}{\widetilde{\color{teal}\text{n}}}$\color{teal}a\color{black}

  \color{teal}(\color{red}862079740\color{teal})\color{black}
\end{center}\vspace{1.618em}

\paragraph{1} Solve:
\begin{cases}
  3x \equiv 3 ($mod $  37)\\
  2x \equiv 15 ($mod $  53)
\end{cases}

\uwave{Slu\enskip.}
\vspace{0.618 em}

$3 \equiv 3 ($mod $37)$ and $3\cdot x \equiv 3\cdot 1 ($mod $37)$ $\implies$ $x
\equiv 1 ($mod $ 37)$

$2\cdotx \equiv 15 \equiv 53 + 15 \equiv 68 \equiv 2\cdot 34 ($mod$
53)$ $\implies$$ 2\cdot x \equiv 2\cdot 34 ($mod $53)$

$2\equiv 2 ($mod $53)$ and $2\cdot x \equiv 2\cdot 34 ($mod $53)$
$\implies x\equiv 34 ($mod $53)$

So, \begin{cases}
  3x \equiv 3 ($mod $  37)\\
  2x \equiv 15 ($mod $  53)
\end{cases}
$=$
\begin{cases}
  x \equiv 1 ($mod $  37)\\
  x \equiv 34 ($mod $  53)
\end{cases}

By the Chinese Remainder Theorem

$x = 1\cdot 53\cdot y_1 + 34\cdot 37\cdot y_2$, where $y_1 = 53^{-1}
($mod $ 37)$ and $y_2 = 37^{-1} ($mod $53)$:\\ $\exists! x^\prime:
x \equiv x^\prime ($mod $34\cdot 53)$ and $0 \leq x^\prime < 34\cdot 53$

$\iff$ gcd$(37,53) = 1$ $\iff$ $37$ and $53$ are relatively prime

So, do the Euclidean Algorithm:

$53 = 1\cdot37 + 16\\
37 = 2\cdot16 + 5\\
16 = 3\cdot5 + 1\\
5 = 5\cdot1 + 0$

We conclude gcd$(37,53) = 1$.

Furthermore, one can now express $1$ as a linear combination of $37$
and $53$ as such:

$16 = 3\cdot5 + 1 \implies 16 - 3\cdot5 = 1\\$
$37 = 2\cdot16 + 5\implies 37 - 2\cdot16 = 5\\$
$53 = 1\cdot37 + 16\implies 53 - 1\cdot37 = 16\\$
$\implies 1 = 53 - 1\cdot37 -3\cdot(37 - 2\cdot(53 - 1\cdot37))$

By counting one can check that $1 = 7\cdot 53 -10\cdot 37$

Immediately $1 \equiv -10\cdot 37 ($mod $53)$ and $1 \equiv 7\cdot 53
($mod $37)$

so, $y_2= -10$ and $y_1 = 7$, from the definition of $y_1$ and $y_2$

Therefore $x = 1\cdot 53\cdot 7 + 34\cdot 37\cdot -10 = -12209$

$7\cdot 37 \cdot 53 -12209 = 1518 \implies -12209 \equiv 1518 ($mod
$37\cdot 53)\\$
and $0 \leq 1518 < 37\cdot 53 = 1961$ $\implies x^\prime = 1518$

So, $\forall t\in \mathbb{Z}: x = t\cdot 1961 +1518$ is a solution of  \begin{cases}
  3x \equiv 3 ($mod $  37)\\
  2x \equiv 15 ($mod $  53)
\end{cases}


\vspace{0.618 em}
$\blacksquare$
\newpage

\paragraph{2} Solve: $345118\cdot x + 6753 y = 1$ $\guillemotright * \guillemotleft$

\uwave{Slu\enskip.}
\vspace{0.618 em}

Do the Euclidean Algorithm to find the initial value $(x_0,y_0)$

$345118 = 51\cdot6753 + 715\\
6753 = 9\cdot715 + 318\\
715 = 2\cdot318 + 79\\
318 = 4\cdot79 + 2\\
79 = 39\cdot2 + 1\\
2 = 2\cdot1 + 0$

One can conclude gcd$(345118,6753) = 1$, and one can say $1 = x\cdot
345118 + y\cdot 6753$ some integers $x,y$:


$345118 = 51\cdot6753 + 715\implies 345118 - 51\cdot6753 = 715\\
6753 = 9\cdot715 + 318\implies 6753 - 9\cdot715 = 318\\
715 = 2\cdot318 + 79\implies 715 - 2\cdot318 = 79\\
318 = 4\cdot79 + 2\implies 318 - 4\cdot79 = 2\\
79 = 39\cdot2 + 1\implies 79 - 39\cdot2 = 1\\
$
$\implies 6753 - 9\cdot(345118 - 51\cdot6753) = 318\\
\implies 345118 - 51\cdot6753 - 2\cdot(6753 - 9\cdot(345118 -
51\cdot6753)) = 79\\
\implies 6753 - 9\cdot(345118 - 51\cdot6753) - 4\cdot(345118 - 51\cdot6753 - 2\cdot(6753 - 9\cdot(345118 -
51\cdot6753))) = 2\\
\implies \\
345118 - 51\cdot6753
- 2\cdot(6753- 9\cdot(345118-51\cdot6753))\\
-39\cdot(6753 - 9\cdot(345118 - 51\cdot6753)
- 4\cdot(345118 - 51\cdot6753 - 2\cdot(6753 - 9\cdot(345118 -
51\cdot6753))))\\ = 1\\
\implies \\
345118 - 51\cdot6753
- 2\cdot 6753 + 18\cdot(345118-51\cdot6753)\\
-39\cdot 6753 + 351 \cdot(345118 - 51\cdot6753)
+ 156 \cdot(345118 - 51\cdot6753 - 2\cdot(6753 - 9\cdot(345118 -
51\cdot6753)))\\ = \\
345118 - 51\cdot6753
- 2\cdot 6753 + 18\cdot 345118 -918 \cdot6753\\
-39\cdot 6753 + 351 \cdot 345118 - 17901 \cdot6753
+ 156 \cdot 345118 - 7956 \cdot6753 - 312\cdot(6753 - 9\cdot(345118 -
51\cdot6753))\\ = \\
$
$345118 - 51\cdot6753
- 2\cdot 6753 + 18\cdot 345118 -918 \cdot6753\\
-39\cdot 6753 + 351 \cdot 345118 - 17901 \cdot6753
+ 156 \cdot 345118 - 7956 \cdot6753 - 312\cdot 6753 + 2808 \cdot(345118 -
51\cdot6753)\\ = \\
345118 - 51\cdot6753
- 2\cdot 6753 + 18\cdot 345118 -918 \cdot6753\\
-39\cdot 6753 + 351 \cdot 345118 - 17901 \cdot6753
+ 156 \cdot 345118 - 7956 \cdot6753 - 312\cdot 6753 + 2808 \cdot 345118 -
143208 \cdot6753\\ = \\
345118 + 18\cdot 345118 + 351 \cdot 345118  + 156 \cdot 345118
+ 2808 \cdot 345118\\
- 51\cdot 6753
- 2\cdot 6753 -918 \cdot 6753
-39\cdot 6753 - 17901 \cdot 6753
- 7956 \cdot6753 - 312\cdot 6753 - 143208 \cdot 6753\\
= \\
(1 + 18 + 351  + 156
+ 2808)\cdot 345118
+(- 51
- 2
-918
-39
- 17901
- 7956
- 312
- 143208) \cdot 6753\\ = 3334\cdot 345118 - 170387\cdot 6753 = 1$

So, a particular solution$(x_0,y_0) = (3334,-170387)$

And by Theorem 9 in the lecture notes:

$\{(x,y): x = 3334 - 6753\cdot t, y = -170387 + 345118\cdot t, t\in
\mathbb{Z}\}$ is the set of all solutions to $\guillemotright * \guillemotleft$

\vspace{0.618 em}
$\blacksquare$
\newpage

\paragraph{3} Prove that if $c$ admits an inverse modulo $m$, then $c$
and $m$ are relatively prime.

\uwave{Pf\enskip.}
\vspace{0.618 em}

Assume $\exists c^{-1}\in \mathbb{Z}: c^{-1}c \equiv 1 ($mod $m)$\\
$\implies$ $\exists t \in \mathbb{Z}: c^{-1}c = mt +1$ (by Theorem 3
(iv) in the lecture notes)\\
$\implies$ $1 = c^{-1}c - tm $\\
$\implies$ $c$ and $m$ are relatively prime (by Corollary 1 in the
lecture notes)

\vspace{0.618 em}
$\blacksquare$

\paragraph{4} In a certain city, mayoral elections occur every 5 years
and last occurred 2 years ago.  Dog-catcher elections, on the other
hand, occur every 7 and occurred last year.  If it is 2019, find the
next year that will feature both mayoral and dog-catcher elections.

\uwave{Slu\enskip.}
\vspace{0.618 em}

Model the problem as a system of congruences

\begin{cases}
  x \equiv 2019-2 $ $($mod $ 5)\\
  x \equiv 2019-1 $ $($mod $ 7)
\end{cases}
$\implies$
\begin{cases}
  x \equiv 2017 \equiv 403\cdot 5 + 2 $ $($mod $ 5)\\
  x \equiv 2018 \equiv 288\cdot 7 + 2 $ $($mod $ 7)
\end{cases}
$\implies$
\begin{cases}
  x \equiv 2 $ $($mod $ 5)\\
  x \equiv 2 $ $($mod $ 7)
\end{cases}


By Chinese Remainder Theorem

$x = 2\cdot 5 \cdot y_1 + 2\cdot 7\cdot y_2$, where $y_1 = 5^{-1}
($mod $7)$, and $y_2 = 7^{-1} ($mod $5)$

Do the Euclidean Algorithm to find $y_1$ and $y_2$

$7 = 1\cdot5 + 2\\
5 = 2\cdot2 + 1\\
2 = 2\cdot1 + 0\\$

$7 = 1\cdot5 + 2\implies 7 - 1\cdot5 = 2\\
5 = 2\cdot2 + 1 \implies 5 - 2\cdot2 = 1\\
\implies 1 = 5 - 2\cdot(7 - 1\cdot5) = 3\cdot5 -2\cdot 7 $

So, $y_1 = 3$ and $y_2 = -2$

Therefore, $x = 2\cdot 5 \cdot 3 + 2\cdot 7\cdot -2 = 2$

Therefore, $x = t\cdot 5 \cdot 7 + 2$ some $t \in \mathbb{Z}$ is a
general solution to the system

Since $2019 = 57\cdot 5\cdot 7 + 24$, and $35$ doesn't divide $24$.

Choosing $t = 57+1$ will give us the answer.

So, $x = 58\cdot 5 \cdot 7 + 2 = 2032$ works.

Because, $2032 = 2017 + 3\cdot 5 = 2018 + 2\cdot7$

\vspace{0.618 em}
$\blacksquare$





\end{document}


%%% Local Variables:
%%% mode: latex
%%% TeX-master: t
%%% End:
